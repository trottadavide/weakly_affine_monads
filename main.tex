\documentclass[a4paper,UKenglish,numberwithinsect,cleveref, autoref, thm-restate]{lipics-v2021}




%\usepackage{natbib}
\usepackage{graphicx}
\usepackage{amsmath}
\usepackage{amsthm}
\usepackage{float}
\usepackage{amsfonts}
\usepackage{amssymb}

\theoremstyle{definition}
\newtheorem*{notation*}{Notation}

%%%%%%%%%%%%%%%%frecce%%%%%%%%%%%%%%


%\newcommand{\relazione}[3]{\xymatrix@-0.8pc{#2 \colon #1  \ar[rr]\ar@{-|}[r]& &  #3}}
\newcommand{\relazione}[3]{#2 \colon #1  \to #3}
\newcommand{\freccia}[3]{#2 \colon #1  \to #3}
\newcommand{\frecciainj}[3]{#2 \colon #1  \to #3}
\newcommand{\frecciasopra}[3]{\xymatrix{ #1  \ar[r]^{#2} &  #3}}
\newcommand{\frecciasopralunga}[3]{\xymatrix{ #1  \ar[rr]^{#2} &&  #3}}
\newcommand{\pbmorph}[2]{#1^{\ast}#2} 
\newcommand{\duefreccia}[3]{\xymatrix@C=0.5cm{#2 \colon #1  \ar@{=>}[r] &  #3}}
%\newcommand{\duefreccianoname}[2]{\xymatrix@C=0.5cm{#1  \ar@{=>}[r] &  #2}}
\newcommand{\duefreccianoname}[2]{#1\leq #2}
\newcommand{\equivalence}[3]{#2 \colon #1 \equiv #3}

\newcommand{\comsquare}[8]{ \xymatrix@+1pc{ 
#1 \ar[r]^{#5} \ar[d]_{#6} & #2 \ar[d]^{#7} \\
#3 \ar[r]_{#8} & #4 
}}
\newcommand{\pullback}[8]{ \xymatrix@+1pc{ 
#1 \pullbackcorner \ar[r]^{#5} \ar[d]_{#6} & #2 \ar[d]^{#7} \\
#3 \ar[r]_{#8} & #4 
}}
\newcommand{\quadratocomm}[8]{ \xymatrix@+1pc{ 
#1 \ar[r]^{#5} \ar[d]_{#6} & #2 \ar[d]^{#7} \\
#3 \ar[r]_{#8} & #4 
}}
\newcommand{\comsquarelargo}[8]{ \xymatrix@+1pc{ 
#1 \ar[rr]^{#5} \ar[d]_{#6} && #2 \ar[d]^{#7} \\
#3 \ar[rr]_{#8} && #4 
}}
\newcommand{\parallelmorphisms}[4]{\xymatrix@+1pc{
#1 \ar @<+4pt>[r]^{#2} \ar @<-4pt>[r]_{#3} & #4
}}
\newcommand{\relation}[4]{\xymatrix@+1pc{
\angbr{#2}{#3}\colon #1 \ar @<+4pt>[r] \ar @<-4pt>[r] & #4
}}
\newcommand{\frecceparalleleopposte}[4]{\xymatrix@+1pc{
#1 \ar@<+4pt>[r]^{#2} \ar@<-4pt>@{<-}[r]_{#3} & #4
}}
\newcommand{\equalizer}[6]{\xymatrix@+1pc{
#1 \ar[r]^{#2} & #3 \ar @<+4pt>[r]^{#4} \ar @<-4pt>[r]_{#5} & #6
}}
\newcommand{\coequalizer}[6]{\xymatrix@+1pc{
 #1 \ar @<+4pt>[r]^{#2} \ar @<-4pt>[r]_{#3} & #4 \ar[r]^{#5} & #6
}}

\newcommand{\subobject}[3]{\xymatrix{
#1 \ar@{>->}[r]^{#2} & #3
}}

\newcommand{\pullbackcorner}[1][ul]{\save*!/#1+1.2pc/#1:(1,-1)@^{|-}\restore}



%%%%%%%%%%%%%%%%%%%%%%%% Categorie %%%%%%%%%
\def\mA{\mathcal{A}}
\def\mB{\mathcal{B}}
\def\mC{\mathcal{C}}
\def\mX{\mathcal{X}}
\def\mD{\mathcal{D}}
\def\mE{\mathcal{E}}
\def\mF{\mathcal{F}}
\def\mM{\mathcal{M}}
\def\mN{\mathcal{N}}
\def\mH{\mathcal{H}}
\def\mK{\mathcal{K}}
\def\mO{\mathcal{O}}
\def\mQ{\mathcal{Q}}
\def\mV{\mathcal{V}}

%%%%%%%%%%%%%%%%%%%%%%%%%%%%%%%%
\def\FinSet{\mathbf{FinSet}}
\def\FinRel{\mathbf{FinRel}}
\def\Set{\mathbf{Set}}
\def\Rel{\mathbf{Rel}}
\def\Preord{\mathbf{Preord}}
\newcommand{\pfn}[1]{#1\mbox{-}\mathbf{Fun}}
\newcommand{\fn}[1]{#1\mbox{-}\mathbf{TFun}}
\newcommand{\total}[1]{#1\mbox{-}\mathbf{Total}}
\newcommand{\alg}[1]{\mathbf{Alg}(#1)}
\newcommand{\Tadj}[2]{#1^{#2}}
%%%%%%%%%%%%%%%%%operatori monoidali%%%%%%%%%%%%%%%%%%%
\def\ox{\otimes}
%%%%%%%%%%%%%%%%%%%%%%%%identità, domini, cod ecc%%%%%%%%5
\def\pr{\operatorname{ pr}}         %projection
\def\id{\operatorname{ id}}         %identity
\def\op{\operatorname{ op}}         %opposite
\def\cod{\operatorname{ cod}}       %codomain
\def\dom{\operatorname{ dom}}      %domain
\def\im{\operatorname{ im}}        %image
\def\ob{\operatorname{ ob}}        %objects
\def\I{I}                          %identity monoidale


%%%%%%%%%%%%%%%%%%%%%%%%%%%%%%%%%%%%%%%%%%%%%%%%%%%%


\newcommand{\angbr}[2]{\langle #1,#2 \rangle} 

\def\gsmcat{\mathbf{GSM\mbox{-}Cat}}

\newcommand{\syntcat}[1]{\mathbf{Th}(#1)}
\newcommand{\context}[3]{[#1_1:#2_1,\dots,#1_{#3}:#2_{#3}]}
\newcommand{\termincontext}[3]{#1:#2 \; [#3]}
\newcommand{\MSalgebra}[1]{\mathrm{#1}}


\usepackage{mathtools}
\usepackage{quiver}
\usepackage{cleveref}
\usepackage{hyperref}
\usepackage{xcolor}
\usepackage[all,2cell]{xy}
\UseAllTwocells
\xyoption{v2}
%%%%%%%%%%%%%%%%%%%%%%%%%%%%%%%%%%%%%%%%%%%%%%%%%%%%%%%
\usepackage{tikz}
\usepackage{tikzit}

\usetikzlibrary{shapes}

%\input{sample.tikzstyle}
\usepackage{freetikz}
\usetikzlibrary{decorations.markings,positioning,patterns}
\usetikzlibrary{shadows}
\usepackage{array}

%\usetikzlibrary{automata, positioning, arrows}
\input{stringdiagrams.tikzstyles}
\usepackage{todonotes}

%%%%%%%%%%%%%%%%%%%%%%%%%%%%%%%%%%%%%
\theoremstyle{plain} %italico
\newtheorem{mytheorem}{Theorem}[section]
\newtheorem{mycorollary}[mytheorem]{Corollary}
\newtheorem{mylemma}[mytheorem]{Lemma}
\newtheorem{myproposition}[mytheorem]{Proposition}


\theoremstyle{definition} %stampatello
\newtheorem{mydefinition}[mytheorem]{Definition}
\newtheorem{myproblem}[mytheorem]{Problem}

\newtheorem{myremark}[mytheorem]{Remark}
\newtheorem{myexample}[mytheorem]{Example}


%%%%%%%%%%%%%%%%%%%%%%%%%%%%%%%%%%%%%


%%%%%%%%%%%%%%%%%%%%%%%%%%%%%%%%%%%

\newcommand{\stringdiagnabla}[1]{\begin{tikzpicture}[scale=0.60, transform shape]
	\begin{pgfonlayer}{nodelayer}
		\node [style=none] (0) at (-4.25, 0) {};
		\node [style=none] (1) at (-2.75, 0.5) {};
		\node [style=none] (2) at (-2.75, -0.5) {};
		\node [style=none] (3) at (-2, -0.5) {};
		\node [style=none] (4) at (-2, 0.5) {};
		\node [style=nodonero] (5) at (-3.25, 0) {};
		\node [style=none] (7) at (-3.75, 0.25) {#1};
	\end{pgfonlayer}
	\begin{pgfonlayer}{edgelayer}
		\draw (0.center) to (5);
		\draw [bend left, looseness=1.25] (5) to (1.center);
		\draw (1.center) to (4.center);
		\draw [bend right] (5) to (2.center);
		\draw (2.center) to (3.center);
	\end{pgfonlayer}
	\pgfsetbaseline{-0.4ex}
\end{tikzpicture}}


\newcommand{\stringdiagbang}[1]{\begin{tikzpicture}[scale=0.60, transform shape]
	\begin{pgfonlayer}{nodelayer}
		\node [style=none] (8) at (2, 0) {};
		\node [style=nodonero] (9) at (3, 0) {};
		\node [style=none] (10) at (2.5, 0.25) {#1};
	\end{pgfonlayer}
	\begin{pgfonlayer}{edgelayer}
		\draw (8.center) to (9);
	\end{pgfonlayer}
		\pgfsetbaseline{-.4ex}
\end{tikzpicture}}

\newcommand{\stringdiagidentity}[1]{\begin{tikzpicture}[scale=0.60, transform shape]
	\begin{pgfonlayer}{nodelayer}
		\node [style=none] (8) at (2, 0) {};
		\node [style=none] (9) at (3, 0) {};
		\node [style=none] (10) at (2.5, 0.25) {#1};
	\end{pgfonlayer}
	\begin{pgfonlayer}{edgelayer}
		\draw (8.center) to (9);
	\end{pgfonlayer}
		\pgfsetbaseline{-.4ex}
\end{tikzpicture}}
\newcommand{\stringdiagfreccia}[3]{\begin{tikzpicture}[scale=0.60, transform shape]
	\begin{pgfonlayer}{nodelayer}
		\node [style=box] (5) at (1.75, 0) {#2};
		\node [style=none] (7) at (3, 0) {};
		\node [style=none] (8) at (2.5, 0.25) {#3};
		\node [style=none] (9) at (0.5, 0) {};
		\node [style=none] (10) at (1, 0.25) {#1};
	\end{pgfonlayer}
	\begin{pgfonlayer}{edgelayer}
		\draw (5) to (7);
		\draw (9.center) to (5);
	\end{pgfonlayer}
	\pgfsetbaseline{-.4ex}
\end{tikzpicture}
}
\newcommand{\nodo}{\begin{tikzpicture}[scale=0.60, transform shape]
	\begin{pgfonlayer}{nodelayer}
		\node [style=nodonero] (0) at (0, 0) {};
	\end{pgfonlayer}
	\pgfsetbaseline{-.4ex}
\end{tikzpicture}}
%%%%%%%%%%%%%%%%%%%%%%%%%%%%%%%%%%%%%%%%


\title{Weakly-affine monads}
\titlerunning{Weakly-affine monads}

\author{Paolo Perrone}{Department of Computer Science, University of Oxford}{paolo.perrone.math@gmail.com}{ }{}
\author{Fabio Gadducci}{Department of Computer Science, University of Pisa, Pisa, IT}{fabio.gadducci@unipi.it}{ https://orcid.org/
0000-0003-0690-3051}{}
\author{ Davide Trotta }{Department of Computer Science, University of Pisa, Pisa, IT}{trottadavide92@gmail.com}{https://orcid.org/0000-0003-4509-594X}{}
\authorrunning{F. Gadducci et alii}
%The research is supported by the MIUR PRIN 2017FTXR "IT-MaTTerS"

\Copyright{Fabio Gadducci and Paolo Perrone and Davide Trotta} 
\ccsdesc[500]{}


\keywords{string diagrams, gs-monoidal categoriess} %TODO mandatory; please add comma-separated list of keywords

\begin{document}

\maketitle

\begin{abstract}
To be written.
\end{abstract}

\section{Weakly Markov categories and weakly affine monads}
\iffalse
\todo[inline]{Se vogliamo generalizzare esempio la proposizione 1.5 da weakly monad su cartesiana a weakly monad su weakly markov, mi sembra che la codizione da chiedere potrebbe essere che T1 sia un gruppo, e che sia ``compatibile'' con la struttura GS. Ho abbazato una possibile definizione }
\begin{mydefinition}
Let $\mC$ be a GS-monoidal category. An internal group $(G,\cdot,I)$ is said to be \textbf{compatible} with the GS-monoidal structure of $\mC$ if every set $\mC(X,G)$ is a group with the following operation, identity element and inverse:
\ctikzfig{group-structure}
 
%\ctikzfig{compatible-group}
\end{mydefinition}
\begin{myremark}
Notice that if $\mC$ is a cartesian monoidal category, every internal group is compatible.
\end{myremark}
\fi
Let $\mC$ be a GS-category. For every object $X$, the set $\mC(X,I)$ 
\todo{How to call them? effects? co-states?}
has a canonical commutative monoid structure as follows: the monoidal unit is the discard map $X\to I$, and given $a,b:X\to I$, their product $ab$ is given by copying, as follows.
\ctikzfig{ptwise-product}
If a morphism $f:X\to Y$ is copyable and discardable, precomposition with $f$ induces a morphism of monoids $\mC(Y,I)\to\mC(X,I)$. 

The monoid $\mC(X,I)$ acts on the set $\mC(X,Y)$: given $a:X\to I$ and $f:X\to Y$, $a\cdot f$ is given as follows,
\ctikzfig{action}
and the product $(f,g)\mapsto f\cdot g \coloneqq (f\otimes g)\circ\mathrm{copy}_X$ is equivariant for this action in both variables. 

\begin{mydefinition}
 A GS-category $\mC$ is called \emph{weakly Markov} if for every object $X$, the monoid $\mC(X,I)$ is a group. 
\end{mydefinition}

Every Markov category is weakly Markov: for each $X$, the monoid $\mC(X,I)$ is the trivial group.

\begin{mydefinition}
 Given two parallel morphisms $f,g:X\to Y$ in a weakly Markov GS-category $\mC$, we say that $f$ and $g$ are \emph{equivalent}, and write $f\sim g$, if they lie in the same orbit for the action of $\mC(X,I)$, i.e.~if there is $a\in \mC(X,I)$ such that $a\cdot f=g$.
 We say they are \emph{uniquely equivalent} if there is a unique $a\in \mC(X,I)$ such that $a\cdot f=g$.
\end{mydefinition}


Let's now consider the case where the GS structure comes from a commutative monad on a cartesian monoidal category $\mD$. 
In this case, the monoid structure of Kleisli morphisms $X\to 1$ comes from the following canonical internal monoid structure of $T1$ in $\mD$, given by 
 \[
 \begin{tikzcd}
  1 \ar{r}{\eta} & T1 ,
 \end{tikzcd}
 \qquad
 \begin{tikzcd}
  T1 \times T1 \ar{r}{c} & T(1\times 1) \ar{r}{\cong} & T1 .
 \end{tikzcd}
 \]
 The monoid structure of Kleisli morphisms $X\to 1$ is now given as follows. The unit is given by
 \[
 \begin{tikzcd}
  X \ar{r}{!} & 1 \ar{r}{\eta} & T1 ,
 \end{tikzcd}
 \]
 and the multiplication of the morphisms $f^\sharp,g^\sharp:X\to T1$ is
 \[
 \begin{tikzcd}
  X \ar{r}{\mathrm{copy}} & X\times X \ar{r}{f^\sharp\times g^\sharp} &
  T1 \times T1 \ar{r}{c} & T(1\times 1) \ar{r}{\cong} & T1 .
 \end{tikzcd}
 \]
 

\begin{mydefinition}
 A commutative monad $T$ on a cartesian monoidal category is called \emph{weakly affine} if $T1$ with its canonical internal monoid structure is a group.
\end{mydefinition}

\begin{myproposition}\label{weaklyboth}
 Let $\mD$ be a cartesian monoidal category, and let $T$ be a commutative monad on $\mD$. The Kleisli category of $T$ is weakly Markov if and only if $T$ is weakly affine.
\end{myproposition}
\begin{proof}
 First, suppose that $T1$ is an internal group, and denote by $\iota:T1\to T1$ its inversion map. 
 The inverse of the morphism $f^\sharp:X\to T1$ in $\mathrm{Kl}_T(X,1)$ is given by $\iota\circ f$: indeed, the following diagram commutes,
 \[
  \begin{tikzcd}
  X \ar{d}[swap]{f^\sharp} \ar{r}{\mathrm{copy}} & X\times X \ar{d}[swap]{f^\sharp\times f^\sharp} \ar{dr}{f\times(\iota\circ f)} \\
  T1 \ar{d}[swap]{!} \ar{r}{\mathrm{copy}} & T1\times T1 \ar{r}[swap]{\id\times\iota} & T1\times T1 \ar{r}{c} & T(1\times 1) \ar{d}{\cong} \\
  1 \ar{rrr}{\eta} &&& T1
  \end{tikzcd}
 \]
 where the bottom rectangle commutes since $\iota$ is the inversion map for $T1$. The analogous diagram with $\iota\times\id$ in place of $\id\times\iota$ commutes analogously.
 
 Conversely, suppose that for every $X$, the monoid structure on $\mathrm{Kl}_T(X,1)$ has inverses. Then in particular we can take $X=T1$, and the inverse of the Kleisli morphism $\id:T1\to T1$ is an inversion map for $T1$. 
\end{proof}

\todo[inline]{This feels vaguely like Yoneda, but in monoidal sauce. Can't make it precise for now.}

\subsection{Examples of weakly affine monads} 

\begin{myexample}
	\label{ex:abelian_group}
	We present a family of examples of commutative monads that are weakly affine but not affine.
	Let $A$ be a commutative monoid (written multiplicatively).
	Then the functor $T_A \coloneqq A \times -$ on $\Set$ has a canonical structure of commutative monad, 
	where the lax structure components $c_{X,Y}$ are given by multiplying elements in $A$ while carrying the elements 
	of $X$ and $Y$ along.

	Since $T_A(1) \cong A$, the monad $T_A$ is weakly affine if and only if $A$ is a group, and affine if and only if $A\cong 1$.
\end{myexample}

\begin{myexample}
    \label{ex:nonzero_measures}
	Let $M^* : \Set \to \Set$ be the monad assigning to every set the set of finitely supported discrete \emph{nonzero}
	measures on $M^*$, or equivalently let $M^*(X)$ for any set $X$ be the set of nonzero finitely supported functions $X \to [0,\infty)$.
	The monad structure is defined in terms of the same formulas as for the distribution monad on $\Set$ and the
	components $c_{X,Y}$ are also given by the formation of product measures, or equivalently point-wise products of functions 
	$X \to [0,\infty)$.

	Since $M^* 1 \cong (0,\infty)\ncong 1$, this monad is not affine. However the monoid structure of $(0,\infty)$ induced by $M^*$ is the usual multiplication of positive real numbers, which form a group. Therefore $M^*$ is weakly affine, and its Kleisli category is weakly Markov.

	On the other hand, if the zero measure is included, then we obtain a commutative monad $M$ which can be seen as the monad of semimodules for the semiring of nonnegative reals.
	Since $M1 \cong [0,\infty)$ is not a group under multiplication, $M$ is not weakly affine.
\end{myexample}

\begin{myexample}
    \label{ex:nonexample}
    Here is a negative example.
    Consider the free abelian group monad $F$ on $\Set$. Its functor takes a set $X$ and forms the set $FX$ of finite multisets (with repetition, where order does not matter) of elements of $X$ and their formal inverses. 
    We have that $F1\cong \mathbb{Z}$, which is an abelian group under addition. 
    However, the monoid structure on $F1$ induced by the monoidal structure of the monad corresponds to the \emph{multiplication} in $\mathbb{Z}$, which does not have inverses. Therefore $F$ is not weakly affine. 
\end{myexample}


\subsection{Conditional independence in weakly Markov categories}

\begin{mydefinition}\label{defcondind}
 A morphism $f:A\to X_1\otimes\dots\otimes X_n$ in a A GS-category $\mC$ is said to exhibit \emph{conditional independence of the $X_i$ given $A$} if and only if it can be expressed as a product of the following form.
 \ctikzfig{cond-ind-sep}
\end{mydefinition}

Note that this is slightly different from \cite[Definition~6.6]{cho_jacobs_2019}, although it is equivalent for the case of Markov categories.

\begin{myproposition}\label{eqcondind}
 Let $f:A\to X_1\otimes\dots\otimes X_n$ be a morphism in a GS-category $\mC$. Then $f$ exhibits conditional independence of the $X_i$ given $A$ if and only if it is equivalent to the product of all its marginals.
 Moreover, in that case $f$ is \emph{uniquely} equivalent to the product of its marginals.
\end{myproposition}

This generalizes the fact that, in Markov categories, a distribution exhibiting conditional independence is the product of its marginals~\cite[Section~12]{Fritz_2020}.

\begin{proof}
 Denote the marginals of $f$ by $f_1,\dots,f_n$.
 Suppose that $f$ is a product as in \Cref{defcondind}. For each $i=1,\dots,n$, by marginalizing, we get that $f_i$ is equal to the following.
  \ctikzfig{cond-ind-proof} 
 Therefore for each $i$ we have that $f_i\sim g_i$. 
 
 Conversely, suppose that $f$ is equivalent to the product of its marginals, i.e.~that there exists $a:X\to I$ such that $f$ is equal to the following.
 \ctikzfig{cond-ind-proof2}
 One can then choose $g_i=f_i$ for all $i<n$, and $g_n = a\cdot f_n$, so that $f$ is in the form of \Cref{defcondind}.
 Moreover, by marginalizing over all the $X_i$ at once, we see that 
 \ctikzfig{cond-ind-proof3}
 so that $a$ is uniquely determined.
\end{proof} 
 
\begin{myremark}
 For $n=2$, a morphism $f:A\to X\otimes Y$ in a weakly Markov GS-category $\mC$ exhibits conditional independence of $X$ and $Y$ given $A$ 
 if and only if the following equation holds.
 \ctikzfig{cond-ind2} 
\end{myremark}


\begin{mylemma}\label{indepkleisli}
 Let $\mD$ be a cartesian monoidal category, and let $T$ be a commutative monad on $\mD$.
 A Kleisli morphism $f^\sharp:A\to T(X_1\times\dots\times X_n)$ exhibits conditional independence of the $X_i$ given $A$ if and only if it factors as follows
 \[
 \begin{tikzcd}
  A \ar{d}[swap]{(g_1^\sharp,\dots,g_n^\sharp)} \ar{dr}{f^\sharp} \\
  TX_1\times\dots\times TX_n \ar{r}[swap]{c} & T(X_1\times\dots\times X_n) ,
 \end{tikzcd}
 \]
 for some Kleisli maps $g_i^\sharp:A\to TX_i$,
 where the map $c$ above is the one obtained by iterating the multiplication of the monoidal structure (such a map is unique by associativity). 
\end{mylemma}
\begin{proof}
 In terms of the base category $\mD$, a Kleisli morphism in the form of \Cref{defcondind} reads as follows.
 \[
  \begin{tikzcd}[sep=large]
   A \ar{r}{\mathrm{copy}} & A\times\dots\times A \ar{r}{g_1^\sharp\times\dots\times g_n^\sharp} & TX_1\times\dots\times TX_n \ar{r}{c} & T(X_1\times\dots\times X_n) .
  \end{tikzcd}
 \]
 Therefore $f^\sharp:A\to T(X_1\times\dots\times X_n)$ is exhibiting conditional independence if and only if it is in the form above.
\end{proof}



\begin{mydefinition}
 Let $\mC$ be a GS-category. We say that $\mC$ satisfies the \emph{localized independence property} if whenever a morphism 
 $f:A\to X\otimes Y\otimes Z$ exhibits conditional independence of $X\otimes Y$ (jointly) and $Z$ given $A$, as well as conditional independence of $X$ and $Y\otimes Z$ given $A$, then it exhibits conditional independence of $X$, $Y$ and $Z$ given $A$. 
\end{mydefinition}


\begin{mytheorem}
 Let $\mD$ be a cartesian monoidal category, and let $T$ be a commutative monad on $\mD$.
 The following conditions are equivalent.
 \begin{enumerate}
  \item\label{condgroup} $T$ is weakly affine;
  \item\label{condwm} $\mathrm{Kl}(T)$ is weakly Markov;
  \item\label{condlocal} $\mathrm{Kl}(T)$ satisfies the localized independence property;
  \item\label{condpullback} For all objects $X$, $Y$, and $Z$, the following associativity diagram is a pullback.
	\begin{equation}
		\label{assoc_pullback}
		\hspace{-9pt}	% T: for centering the equation number
		\begin{tikzcd}[column sep=3.3pc]
			T(X) \times T(Y) \times T(Z) \ar{r}{\id \times c_{Y,Z}} \ar[swap]{d}{c_{X,Y} \times \id}	& T(X) \times T(Y \times Z) \ar{d}{c_{X,Y \times Z}}	\\
			T(X \times Y) \times T(Z) \ar{r}{c_{X\times Y,Z}}						& T(X \times Y \times Z)
		\end{tikzcd}
	\end{equation}
 \end{enumerate}
\end{mytheorem}

\begin{proof}
 $\ref{condgroup}\Leftrightarrow\ref{condwm}$: 
 see \Cref{weaklyboth}.
 
 $\ref{condwm}\Rightarrow\ref{condlocal}$: 
 Suppose $f:A\to X\otimes Y\otimes Z$ exhibits conditional independence of $X\otimes Y$ (jointly) and $Z$ given $A$, as well as conditional independence of $X$ and $Y\otimes Z$ given $A$
 By marginalizing out $X$, we have that $f_{YZ}$ exhibits conditional independence of $Y$ and $Z$ given $A$. 
 Since by hypothesis $f$ exhibits conditional independence of $X$ and $Y\otimes Z$ given $A$, by \Cref{eqcondind} we have that $f$ is equivalent to the product  of $f_X$ and $f_{YZ}$. But, again by \Cref{eqcondind}, $f_{YZ}$ is equivalent to the product of $f_Y$ and $f_Z$, so we have that $f$ is equivalent to the product of all its marginals. Using \Cref{eqcondind} in the other direction, thie means that $f$ exhibits conditional independence of $X$, $Y$ and $Z$ given $A$. 
 
 $\ref{condlocal}\Rightarrow\ref{condpullback}$: 
 By the universal property of products, a cone over the cospan in \eqref{assoc_pullback} consists of maps $g_1^\sharp:A\to TX$, $g_{23}^\sharp:A\to T(Y\times Z)$, $g_{12}^\sharp:A\to T(X\times Y)$ and $g_3^\sharp:A\to TZ$ such that the following diagram commutes. 
 \[
  \begin{tikzcd}[column sep=3.3pc]
   A \ar[bend left=10]{drr}{(g_1^\sharp,g_{23}^\sharp)} \ar[bend right=15]{ddr}[swap]{(g_{12}^\sharp,g_3^\sharp)} \\
   & T(X) \times T(Y) \times T(Z) \ar{r}[swap]{\id \times c_{Y,Z}} \ar{d}{c_{X,Y} \times \id}	& T(X) \times T(Y \times Z) \ar{d}{c_{X,Y \times Z}}	\\
   & T(X \times Y) \times T(Z) \ar{r}[swap]{c_{X\times Y,Z}}						& T(X \times Y \times Z)
  \end{tikzcd}
 \]
 By \Cref{indepkleisli}, this amounts to a Kleisli map $f^\sharp:A\to T(X\times Y\times Z)$ exhibiting conditional independence of $X$ and $Y\otimes Z$ given $A$, as well as of $X\otimes Y$ and $Z$ given $A$. By the localized independence property, we then have that $f$ exhibits conditional independence of all $X$, $Y$ and $Z$ given $A$, and so, again by \Cref{indepkleisli}, $f^\sharp$ factors through the product $TX\times TY\times TZ$. 
 More specifically, by marginalizing over $Z$, we have that $g_{12}^\sharp$ factors through $TX\times TY$, i.e.~the following diagram on the left commutes for some $h_1^\sharp:A\to TX$ and $h_2^\sharp:A\to TY$, and similarly, by marginalizing over $X$, the diagram on the right commutes for some $\ell_2^\sharp:A\to TY$ and $\ell_3^\sharp:A\to TZ$.
 \[
  \begin{tikzcd}
   A \ar{d}[swap]{(h_1^\sharp,h_2^\sharp)} \ar{dr}{g_{12}^\sharp} \\
   TX\times TY \ar{r}[swap]{c} & T(X\times Y) 
  \end{tikzcd}
  \qquad
  \begin{tikzcd}
   A \ar{d}[swap]{(\ell_2^\sharp,\ell_3^\sharp)} \ar{dr}{g_{23}^\sharp} \\
   TY\times TZ \ar{r}[swap]{c} & T(Y\times Z) 
  \end{tikzcd}
 \]
 In other words, the upper and the left curved triangles in the following diagram commute.
\[
  \begin{tikzcd}[column sep=3.3pc]
   A \ar[bend left=15, shift left]{drr}{(g_1^\sharp,g_{23}^\sharp)} \ar[bend right=35]{ddr}[swap]{(g_{12}^\sharp,g_3^\sharp)} 
    \ar[shift left]{dr}[near end]{(g_1^\sharp,\ell_2^\sharp,\ell_3^\sharp)} \ar[shift right]{dr}[swap, pos=0.7]{(h_1^\sharp,h_2^\sharp, g_3^\sharp)} \\
   & T(X) \times T(Y) \times T(Z) \ar{r}[swap]{\id \times c_{Y,Z}} \ar{d}{c_{X,Y} \times \id}	& T(X) \times T(Y \times Z) \ar{d}{c_{X,Y \times Z}}	\\
   & T(X \times Y) \times T(Z) \ar{r}[swap]{c_{X\times Y,Z}}						& T(X \times Y \times Z)
  \end{tikzcd}
 \]
 By marginalizing over $Y$ and $Z$, there exists a unique $a^\sharp:A\to T1$ such that $h_1 = a\cdot g_1$. 
 Therefore
 \[
  g_{12} = h_1\cdot h_2 = (a\cdot g_1) \cdot h_2 = g_1\cdot (a\cdot h_2) ,
 \]
 and so in the diagram above we can equivalently replace $h_1$ and $h_2$ with $g_1$ and $a\cdot h_2$.
 Similarly by marginalizing over $X$ and $Y$, there exists a unique $c^\sharp:A\to T1$ such that $\ell_3=c\cdot g_3$, so that
 \[
  g_{23}= \ell_2\cdot\ell_3 = \ell_2\cdot (c\cdot g_3) = (c\cdot \ell_2) \cdot g_3
 \]
 and in the diagram above we can replace $\ell_2$ and $\ell_3$ with $c\cdot \ell_2$ and $g_3$, as follows.
 \[
  \begin{tikzcd}[column sep=3.3pc]
   A \ar[bend left=15, shift left]{drrr}{(g_1^\sharp,g_{23}^\sharp)} \ar[bend right=35]{ddrr}[swap]{(g_{12}^\sharp,g_3^\sharp)} 
    \ar[shift left]{drr}[near end]{(g_1^\sharp,(c\cdot\ell_2)^\sharp,g_3^\sharp)} \ar[shift right]{drr}[swap, pos=0.8]{(g_1^\sharp,(a\cdot h_2)^\sharp, g_3^\sharp)} \\
   && T(X) \times T(Y) \times T(Z) \ar{r}[swap]{\id \times c_{Y,Z}} \ar{d}{c_{X,Y} \times \id}	& T(X) \times T(Y \times Z) \ar{d}{c_{X,Y \times Z}}	\\
   && T(X \times Y) \times T(Z) \ar{r}[swap]{c_{X\times Y,Z}}						& T(X \times Y \times Z)
  \end{tikzcd}
 \]
 Now, marginalizing over $Z$ and $Z$, we see that necessarily $a\cdot h_2=c\cdot \ell_2$. 
 Therefore there is a unique map $A\to TX\times TY\times TZ$ making the whole diagram commute, which means that \eqref{assoc_pullback} is a pullback.
 
 $\ref{condpullback}\Rightarrow\ref{condgroup}$:
 If $T$ is weakly affine, then taking $X = Y = Z = 1$ in~\eqref{assoc_pullback} shows that this monoid must be an abelian group: we obtain a unique arrow $\freccia{T(1)}{\iota}{T(1)}$ making the following diagram commute,
	\[
		\begin{tikzcd}
			T1 \ar{dr}[description]{(\id,\iota,\id)} \ar[bend left=15]{drr}{(\id,\eta_1 !)} \ar[bend right,swap]{ddr}{(\eta_1 !,\id)} \\
			&	T1 \times T1 \times T1 \ar{r}{\id \times c_{1,1}} \ar[swap]{d}{c_{1,1} \times \id}	& T1 \times T(1\times 1) \ar{d}{c_{1,1\times 1}} \ar{r}{\cong} & T1\times T1 \ar{d}{c_{1,1}}	\\
			&	T(1\times 1) \times T1 \ar{r}{c_{1\times 1,1}} \ar{d}[swap]{\cong}	& T(1\times 1\times 1) \ar{d}{\cong} \ar{r}{\cong} & T(1\times 1) \ar{d}{\cong} \\
			& T1\times T1 \ar{r}[swap]{c_{1,1}} & T(1\times 1) \ar{r}[swap]{\cong} & T1
		\end{tikzcd}
	\]
	and the commutativity shows that $\iota$ satisfies the equations making it the inversion map for a group structure.
\end{proof}



% For context:
% \begin{myproposition}
%  A monoid $(M,\cdot,1)$ is a group if and only if the associativity square
%  \begin{equation}\label{assoc}
%   \begin{tikzcd}
%    M \times M \times M \ar{d}{\id\times\cdot} \ar{r}{\cdot\times\id} & M\times M \ar{d}{\cdot} \\
%    M\times M \ar{r}{\cdot} & M
%   \end{tikzcd}
%  \end{equation}
%  is a pullback.
% \end{myproposition}
% \begin{proof}
%  The square~\eqref{assoc} is a pullback, both of sets and of groups, if and only if given $a,g,h,c\in M$ such that $ag=hc$, there exists a unique $b\in M$ such that $g=bc$ and $h=ab$.
%  First, suppose that $g$ is a group. The only possible choice of $b$ is 
%  \[
%   b = a^{-1}h = gc^{-1},
%  \]
%  which is unique by uniqueness of inverses. 
%  
%  Conversely, suppose that \eqref{assoc} is a pullback. We can set $g,h=e$ and $c=a$ so that $ae=ea=a$. 
%  Instantiating the pullback property, there is a unique $b$ such that $ab=e$ and $ba=e$, that is, $b=a^{-1}$.
% \end{proof}


\section{Additional material (to be added to section)}

\begin{myproposition}\label{prop:actio_group_as_pullback}
 Let $(G,\cdot,1)$ be a group and let $X$ be a set. A function $\alpha: M\times X\to X$ determines a left action if and only if the square
 \begin{equation}\label{action}
  \begin{tikzcd}
   G \times G \times X \ar{d}{\id\times\alpha} \ar{r}{\cdot\times\id} & M\times X \ar{d}{\alpha} \\
   G\times X \ar{r}{\alpha} & X
  \end{tikzcd}
 \end{equation}
 commutes and it is a pullback.
\end{myproposition}
\begin{proof}
By definition, the square \eqref{action} commutes if and only if $\alpha$ and $\cdot$ are compatible. Now we show that the commutative square \eqref{action} is a pullback if and only if $\alpha$ satisfies the identity axiom, i.e. $\alpha(e,x)=x$ for every $x$ in $X$.
Now, if \eqref{action} is a pullback, then there exists a function $\beta:X\to G$ such that the diagram
% https://q.uiver.app/?q=WzAsNSxbMSwxLCJNXFx0aW1lcyBNXFx0aW1lcyBYIl0sWzMsMSwiTVxcdGltZXMgWCJdLFsxLDMsIk1cXHRpbWVzIFgiXSxbMywzLCJYIl0sWzAsMCwiIFgiXSxbMCwxLCJpZFxcdGltZXMgXFxhbHBoYSJdLFswLDIsIlxcY2RvdFxcdGltZXMgaWQiLDJdLFsyLDMsIlxcYWxwaGEiLDJdLFsxLDMsIlxcYWxwaGEiXSxbNCwyLCJlISxpZCIsMix7ImN1cnZlIjo1fV0sWzQsMSwiZSEsaWQiLDAseyJjdXJ2ZSI6LTV9XSxbNCwwLCJlISxcXGJldGEsaWQiLDFdXQ==
\[\begin{tikzcd}
	{ X} \\
	& {G\times G\times X} && {G\times X} \\
	\\
	& {G\times X} && X
	\arrow["{id\times \alpha}", from=2-2, to=2-4]
	\arrow["{\cdot\times id}"', from=2-2, to=4-2]
	\arrow["\alpha"', from=4-2, to=4-4]
	\arrow["\alpha", from=2-4, to=4-4]
	\arrow["\angbr{e!}{\id}"', curve={height=30pt}, from=1-1, to=4-2]
	\arrow["\angbr{e!}{\id}", curve={height=-30pt}, from=1-1, to=2-4]
	\arrow["\angbr{e!}{\beta,\id}"{description}, from=1-1, to=2-2]
\end{tikzcd}\]
commutes, where $e!:X\to G$ is the function assigning the identity element $e$ to every element $x$ of $X$. Now, since the left triangle commutes, then we have that $e=e\cdot \beta (x)$ for every $x$ of $X$, i.e. $\beta(x)=e$ for every $x$ of $X$. Now, since the right triangle commutes, we can conclude that $\alpha (\beta(x),x)=\alpha (e,x)=x$ for every $x$ in $X$. 

Now we show that $\alpha (e,x)=x$ implies that  the commutative square \eqref{action} is a pullback. Let us consider a set $Y$ and the functions $\angbr{f_1}{f_2}:Y\to G\times X$ and  $\angbr{g_1}{g_2}:Y\to G\times X$ such that $\alpha( f_1(y),f_2(y))=\alpha( g_1(y),g_2(y))$. By applying $\alpha (f_1(y)^{-1},-)$ to both sides, and then combining the compatibility of $\alpha$ with the assumption that $\alpha (e,x)=x$, we can conclude that $f_2(y)=\alpha (f_1(y)^{-1}\cdot g_1(y),g_2(y))$.
%similarly, that $g_2(y)=\alpha (g_1(y)^{-1}\cdot f_1(y),f_2(y))$. 
Therefore, we can conclude that the diagram 
% https://q.uiver.app/?q=WzAsNSxbMSwxLCJNXFx0aW1lcyBNXFx0aW1lcyBYIl0sWzMsMSwiTVxcdGltZXMgWCJdLFsxLDMsIk1cXHRpbWVzIFgiXSxbMywzLCJYIl0sWzAsMCwiIFkiXSxbMCwxLCJpZFxcdGltZXMgXFxhbHBoYSJdLFswLDIsIlxcY2RvdFxcdGltZXMgaWQiLDJdLFsyLDMsIlxcYWxwaGEiLDJdLFsxLDMsIlxcYWxwaGEiXSxbNCwyLCJnXzEuZ18yIiwyLHsiY3VydmUiOjV9XSxbNCwxLCJmXzEsZl8yIiwwLHsiY3VydmUiOi01fV0sWzQsMCwiZl8xLGZfMV57LTF9Z18xLGdfMiIsMV1d
\[\begin{tikzcd}
	{ Y} \\
	& {M\times M\times X} && {M\times X} \\
	\\
	& {M\times X} && X
	\arrow["{id\times \alpha}", from=2-2, to=2-4]
	\arrow["{\cdot\times id}"', from=2-2, to=4-2]
	\arrow["\alpha"', from=4-2, to=4-4]
	\arrow["\alpha", from=2-4, to=4-4]
	\arrow["\angbr{g_1}{g_2}"', curve={height=30pt}, from=1-1, to=4-2]
	\arrow["\angbr{f_1}{f_2}", curve={height=-30pt}, from=1-1, to=2-4]
	\arrow["\angbr{f_1}{\gamma,g_2}"{description}, from=1-1, to=2-2]
\end{tikzcd}\]
commutes, where the function $\gamma: Y \to M$ is defined by $\gamma (y):= f_1^{-1}(y)\cdot g_1 (y)$. By the unicity of the inverse in a group, this function is also unique, and hence we can conclude that the commutative square \eqref{action} is a pullback.
\end{proof}
\section{Weakly-affine monads}
\todo{Nome da scegliere e valutare se dare la def per una arbitraria gs}
\begin{mydefinition}
Let $T$ be a commutative monad on a category $\mA$ with finite products. A triple $(X,Y,Z)$ of objects of $\mA$ is said to be \textbf{TBA} if the commutative square 
\[\begin{tikzcd}[column sep=3.3pc]
			T(X) \times T(Y) \times T(Z) \ar{r}{\id \times c_{Y,Z}} \ar[swap]{d}{c_{X,Y} \times \id}	& T(X) \times T(Y \times Z) \ar{d}{c_{X,Y \times Z}}	\\
			T(X \times Y) \times T(Z) \ar{r}{c_{X\times Y,Z}}						& T(X \times Y \times Z)
		\end{tikzcd}
		\]
		is a pullback.
\end{mydefinition}

\todo{esempi?}
\begin{mydefinition}
	Let $T$ be a commutative monad on a category $\mA$ with finite products. 
	We say that the monad $T$ is \textbf{weakly affine} if the
	following associativity diagram is a pullback for every $X,Y,Z$ in $\mA$:
	\begin{equation}
		\label{c_assoc_pullback}
		\hspace{-9pt}	% T: for centering the equation number
		\begin{tikzcd}[column sep=3.3pc]
			T(X) \times T(Y) \times T(Z) \ar{r}{\id \times c_{Y,Z}} \ar[swap]{d}{c_{X,Y} \times \id}	& T(X) \times T(Y \times Z) \ar{d}{c_{X,Y \times Z}}	\\
			T(X \times Y) \times T(Z) \ar{r}{c_{X\times Y,Z}}						& T(X \times Y \times Z)
		\end{tikzcd}
	\end{equation}
\end{mydefinition}

Let us derive some general properties before looking at examples.
It is a standard fact that for any commutative monad $T$, the composite arrow
\[
	\begin{tikzcd}
		T(1) \times T(1) \ar{r}{c_{1,1}}	& T(1 \times 1) \ar{r}{\cong}	& T(1)		
	\end{tikzcd}
\]
equips $T(1)$ with the structure of a commutative monoid internal to $\mathcal{A}$ with unit $\freccia{1}{\eta_1}{T(1)}$.

\begin{mylemma}\label{lem_T1_group}
	\label{lem:T1_group}
	If $T$ is weakly affine, then $T(1)$ is a group.
\end{mylemma}

%\tob{It's possible to generalize the definition of weakly affine monad to the case where $\mA$ is merely gs-monoidal or even just symmetric monoidal. However, I have not implemented this (yet) since already the proof of this lemma does not generalize straightforwardly}

\begin{proof}
	If $T$ is weakly affine, then taking $X = Y = Z = 1$ in~\eqref{c_assoc_pullback} shows that this monoid must be an abelian group:
	assuming that $\times$ is a strict monoidal structure for simplicity, we obtain a unique arrow $\freccia{T(1)}{\iota}{T(1)}$ such that the diagram
	\[
		\begin{tikzcd}
			T(1) \ar{dr}[description]{(\id,\iota,\id)} \ar[bend left]{drr}{(\id,\eta_1 !)} \ar[bend right,swap]{ddr}{(\eta_1 !,\id)} \\
			&	T(1) \times T(1) \times T(1) \ar{r}{\id \times c_{1,1}} \ar[swap]{d}{c_{1,1} \times \id}	& T(1) \times T(1) \ar{d}{c_{1,1}}	\\
			&	T(1) \times T(1) \ar{r}{c_{1,1}}								& T(1)
		\end{tikzcd}
	\]
	and the commutativity shows that $\iota$ satisfies the equations making it the inversion map for a group structure.
\end{proof}

\begin{myproposition}
	\label{lem:group_action}
	If $T$ is weakly affine, then for every object $X$, the morphism $c_{1,X}: T(1)\times T(X)\to T(X)$ determines a (left) group action. 

\end{myproposition}
\begin{proof}
%Let $\freccia{T(X)}{\gamma}{T(X)}$ be the unique morphism such that the diagram
%	\[
%		\begin{tikzcd}
%			T(X) \ar{dr}[description]{(\eta_1!,\gamma,\eta_1!)} \ar[bend left]{drr}{(\eta_1!,\id)} \ar[bend right,swap]{ddr}{(\id,\eta_1!)} \\
%			&	T(1) \times T(X) \times T(1) \ar{r}{\id \times c_{X,1}} \ar[swap]{d}{c_{1,X} \times \id}	& T(1) \times T(X) \ar{d}{c_{1,X}}	\\
%			&	T(X) \times T(1) \ar{r}{c_{X,1}}								& T(X)
%		\end{tikzcd}
%	\]
%	commutes. Now we show that the morphism $c_{1,X}\circ (\id_{T1}\times \gamma): T(1)\times T(X)\to T(X)$ provides an action of the group $T(1)$ on $T(X)$. First, notice that identity axiom, i.e. $(c_{1,X} (\id_{T1}\times \gamma))(\eta_1!, \id)=\id$, is satisfied by definition of $\gamma$, because $c_{1,X}(\eta_1!,\gamma)=\id$. 
	The compatibility axiom follows from the fact that the diagram
	% https://q.uiver.app/?q=WzAsNCxbMCwwLCJUKDEpXFx0aW1lcyBUKDEpXFx0aW1lcyBUKFgpIl0sWzIsMCwiVCgxKVxcdGltZXMgVChYKSJdLFsyLDEsIlQoWCkiXSxbMCwxLCJUKDEpXFx0aW1lcyBUKFgpIl0sWzAsMSwiaWRcXHRpbWVzIGNfezEsWH0iXSxbMSwyLCJjX3sxLFh9Il0sWzAsMywiY197MSwxfVxcdGltZXMgaWQiLDJdLFszLDIsImNfezEsWH0iLDJdXQ==
\[\begin{tikzcd}
	{T(1)\times T(1)\times T(X)} && {T(1)\times T(X)} \\
	{T(1)\times T(X)} && {T(X)}
	\arrow["{\id\times c_{1,X}}", from=1-1, to=1-3]
	\arrow["{c_{1,X}}", from=1-3, to=2-3]
	\arrow["{c_{1,1}\times \id}"', from=1-1, to=2-1]
	\arrow["{c_{1,X}}"', from=2-1, to=2-3]
\end{tikzcd}\]
commutes for every strong and commutative monad. Moreover, following the same proof used for \Cref{prop:actio_group_as_pullback}, we can conclude that the identity axiom is satisfied since $T$ is weakly affine. In particular, because $T(1)$ is a group by \Cref{lem_T1_group}, and the previous square is a pullbackc (by definition of weakly affine monad).
\end{proof}


\begin{myproposition}
Let $T$ be a weakly affine monad. If the diagram
% https://q.uiver.app/?q=WzAsNCxbMCwwLCJUKDEpIl0sWzAsMSwiVCgxKSJdLFsxLDEsIlReMigxKSJdLFsxLDAsIlQoMSkiXSxbMCwxLCJcXGlvdGEiLDJdLFsxLDIsIlQoXFxldGFfMSkiLDJdLFswLDNdLFszLDIsIlxcZXRhX3tUMX0iXV0=
\[\begin{tikzcd}
	{T(1)} & {T(1)} \\
	{T(1)} & {T^2(1)}
	\arrow["\iota"', from=1-1, to=2-1]
	\arrow["{T(\eta_1)}"', from=2-1, to=2-2]
	\arrow["\id",from=1-1, to=1-2]
	\arrow["{\eta_{T1}}", from=1-2, to=2-2]
\end{tikzcd}\]
commutes, then $T^2(1)\cong T(1)$ in $\mA$. 

\end{myproposition}
\begin{proof}
	To prove the result it is enough to show that $T(1)\cong 1$ in the Kleisli category $\mA_T$.
We know from Lemma~\label{lem:T1_group} that $T(1)$ is a group in $\mA$, where the arrow $\freccia{1}{\eta_1}{T(1)}$ is the unit of the group, and $\freccia{T(1)}{\iota}{T(1)}$ is the inversion map. Therefore, we have that the composition $\freccia{1}{\iota\eta_1}{T(1)}$ has to be equal to $\eta_1$. Therefore, we can consider the arrows $1\to T(1)$ and $T(1)\to 1$ in the Kleisli category $\mA_T$ given by $T(\eta_1)\eta_1$ and $\iota$ respectively. The composition $T(\eta_1)\eta_1$ with $\iota$ in $\mA_T$ is given by $\mu_{1,1}T(\iota)T(\eta_1)\eta_1$. Employing the naturality of $\eta_1$ and the fact that $\iota\eta_1=\eta_1$, it is direct to check that $\mu T(\iota)T(\eta_1)\eta_1=\eta_1$, that is the identity $1\to 1$ in $\mA_T$. Now to show that the other composition gives the identity on $T(1)$ in $\mA_T$, it is enough to show that $T(\eta_1)\iota=\eta_{T(1)}$, but this follows by hypothesis.
\end{proof}
\todo[inline]{(Paolo) Credo che $T(\eta_1)\iota\ne\eta_{T(1)}$ nell'esempio delle misure nonzero. Per ogni $x$ in $(0,\infty)=T1$ abbiamo che $\eta_{T(1)}(x)=\delta_x$ (delta di Dirac), mentre $T\eta_1(\iota(x)) = T\eta_1(1/x) = 1/x\,\delta_1$.}
\begin{mycorollary}
	Let $T$ be a weakly affine monad. If the diagram
	% https://q.uiver.app/?q=WzAsNCxbMCwwLCJUKDEpIl0sWzAsMSwiVCgxKSJdLFsxLDEsIlReMigxKSJdLFsxLDAsIlQoMSkiXSxbMCwxLCJcXGlvdGEiLDJdLFsxLDIsIlQoXFxldGFfMSkiLDJdLFswLDNdLFszLDIsIlxcZXRhX3tUMX0iXV0=
	\[\begin{tikzcd}
		{T(1)} & {T(1)} \\
		{T(1)} & {T^2(1)}
		\arrow["\iota"', from=1-1, to=2-1]
		\arrow["{T(\eta_1)}"', from=2-1, to=2-2]
		\arrow["\id",from=1-1, to=1-2]
		\arrow["{\eta_{T1}}", from=1-2, to=2-2]
	\end{tikzcd}\]
	commutes, then $T(1)$ is an idempotent group, namely $\iota=\id_{T1}$.
\end{mycorollary}
\begin{proof}
By \Cref{lem_T1_group} we have that $T(1)$ is a group. If $\eta_{T1}=T(\eta_{1})\iota$, then  we can apply the multiplication of the monad to both sides, obtaining $\iota=\id_{T1}$.
\end{proof}
The following result shows that weak affinity occurs frequently. 
Recall that a strong monad $\freccia{\mA}{T}{\mA}$ on a category $\mA$ with finite products is \textbf{affine} if $T(1)\cong 1$ (see also Remark~\ref{rem:affine monad}). Three relevant examples of affine monads are the distribution monad on $\Set$ (for discrete probability), the Giry monad on the category of measurable spaces (for measure-theoretic probability, see Examples~\ref{ex: Giry monad} and~\ref{ex:Stoch and QBS_P are gs-monoidal}), and the expectation monad, see~\cite{Jacobs16}.

\begin{myproposition}\label{prop:every affine commutative monad is weakly affine}
	Let $T$ be a commutative monad on a category $\mA$ with finite limits. 
	If $T$ is affine, then it is weakly affine.
\end{myproposition}


\begin{proof}
	Let $m_{X,Y} : T(X \times Y) \longrightarrow TX \times TY$
	be the arrow defined as the pairing of $T(\pi_1)$ and $T(\pi_2)$.
	Then it is known that $T$ is affine if and only if $m_{X,Y} c_{X,Y} = \id_{TX \times TY}$~\cite[Lemma~4.2(i)]{Jacobs1994}.\footnote{For probability monads, this equation can be interpreted as stating that the marginals of a product distribution are the original factors~\cite{Fritz2018}.}
	In particular, $c_{X,Y}$ is a split mono and therefore mono.

	To show that~\eqref{c_assoc_pullback} is a pullback, we prove the universal property starting with a diagram
	\begin{equation}
		\label{candidate_pullback}
		\begin{tikzcd}[column sep=3pc]
			A \ar[bend left]{drr}{(f_1, f_2)} \ar[bend right,swap]{ddr}{(g_1, g_2)} \ar[dashed]{dr}{\exists!}							\\
				& TX \times TY \times TZ \ar{r}{\id \times c_{Y,Z}} \ar[swap]{d}{c_{X,Y} \times \id}	& TX \times T(Y \times Z) \ar{d}{c_{X,Y \times Z}}	\\
				& T(X \times Y) \times TZ \ar{r}{c_{X\times Y,Z}}						& T(X \times Y \times Z)
		\end{tikzcd}
	\end{equation}
	where the dashed arrow will be constructed; its uniqueness is clear since $\id \times c_{Y,Z}$ and $c_{X,Y} \times \id$ are mono, so it remains to prove existence.
	Taking the unlabelled arrows to be (induced by) product projections, we have the commutative diagram
	\iffalse
	\[
		\begin{tikzcd}[column sep=small]
			A \ar{r}{(g_1, g_2)} \ar[swap]{d}{(f_1, f_2)}					& T(X \times Y) \times TZ \ar{r}{c_{X \times Y, Z}} 	& T(X \times Y \times Z) \ar{d}	\\
			TX \times T(Y \times Z) \ar{rr}	\ar["c_{X,Y \times Z}" description]{urr} 	&							& T(Y \times Z)
		\end{tikzcd}
	\]
	\fi
% https://q.uiver.app/?q=WzAsNSxbMCwwLCJBIl0sWzEsMCwiVChYXFx0aW1lcyBZKVxcdGltZXMgVFoiXSxbMCwxLCJUWFxcdGltZXMgVChZXFx0aW1lcyBaKSJdLFszLDAsIlQoWFxcdGltZXMgWVxcdGltZXMgWikiXSxbMywxLCJUKFlcXHRpbWVzIFopIl0sWzAsMSwiKGdfMSxnXzIpIl0sWzAsMiwiKGZfMSxmXzIpIiwyXSxbMyw0XSxbMiw0XSxbMiwzLCJjX3tYLFlcXHRpbWVzIFp9IiwxXSxbMSwzXV0=
\[\begin{tikzcd}[column sep=tiny]
	A & {T(X\times Y)\times TZ} && {T(X\times Y\times Z)} \\
	{TX\times T(Y\times Z)} &&& {T(Y\times Z)}
	\arrow["{(g_1,g_2)}", from=1-1, to=1-2]
	\arrow["{(f_1,f_2)}"', from=1-1, to=2-1]
	\arrow[from=1-4, to=2-4]
	\arrow[from=2-1, to=2-4]
	\arrow["{c_{X,Y\times Z}}"{description}, from=2-1, to=1-4]
	\arrow[from=1-2, to=1-4]
\end{tikzcd}\]
	where the upper left triangle commutes by assumption, and the lower right triangle commutes by naturality of $c$ with respect to the unique arrow $X \to 1$ together with $T1 \cong 1$ and the fact that $c_{1,Y \times Z}$ is a coherence isomorphism.
	By the naturality of $c$, $f_2$ can be written as the composite
	\[
		\begin{tikzcd}[column sep=scriptsize]
			A \ar{r}{(g_1, g_2)}	& T(X \times Y) \times TZ \ar{r}	& TY \times TZ \ar{r}{c_{Y,Z}}	& T(Y \times Z).			
		\end{tikzcd}
	\]
	By analogous reasoning, we identify $g_1$ with the composite
	\[
		\begin{tikzcd}[column sep=scriptsize]
			A \ar{r}{(f_1, f_2)}	& TX \times T(Y \times Z) \ar{r}	& TX \times TY \ar{r}{c_{X,Y}}	& T(X \times Y).			
		\end{tikzcd}
	\]
	Getting back to~\eqref{candidate_pullback}, we take the dashed arrow to be the arrow whose component on $TX$ is given by $f_1$, on $TZ$ by $g_2$, and on $TY$ by the diagonal in the diagram
	\[
		\begin{tikzcd}
			A \ar{r}{f_2} \ar[swap]{d}{g_1}	& T(Y \times Z)	\ar{d}	\\
			T(X \times Y) \ar{r}		& TY
		\end{tikzcd}
	\]
	which commutes for similar reasons as above.
	The fact that this arrow recovers the $f_2$ component after composition with $\id \times c_{Y,Z}$ and the $g_1$ component after composition with $c_{X,Y} \times \id$ follows by the expressions for $f_2$ and $g_1$ derived above.
	The fact that it recovers $f_1$ and $g_2$ is by construction.

\end{proof}

\begin{myremark}
	We are not aware of any relation between weakly affine monads in our sense and Jacobs' \emph{strongly affine} monads~\cite{Jacobs16}, other than the fact that strongly affine implies affine implies weakly affine.
\end{myremark}






\end{document}


